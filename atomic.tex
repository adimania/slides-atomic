% ansible.tex
\documentclass{beamer}
\usetheme{Warsaw}
\author{Aditya Patawari}
\institute{
  Contributor to Fedora Admin team\\[2ex]
  Lead Engineer (Infrastructure) at BrowserStack.com\\[2ex]
  \texttt{aditya@adityapatawari.com}\\
  \texttt{adimania on freenode irc}\\
  \texttt{http://blog.adityapatawari.com}
}
\title{Running your containers in a sane environment, Project Atomic}
\usepackage{graphicx}
\usepackage{ragged2e}
\usepackage{fancyvrb}
\usepackage[free-standing-units, space-before-unit, use-xspace]{siunitx}
\begin{document}

\begin{frame}
  \titlepage
  \end{frame}
\begin{frame}{Topics}
\begin{itemize}
  \item What is the problem?
  \item Project Atomic is here!
  \item .. Along with some components
  \item Starting our Atomic Host
  \item Demo
\end{itemize}
\end{frame}

\begin{frame}{What is the problem?}
\begin{figure}[htp]
\centering
\includegraphics[scale=0.15]{problem.jpg}
\label{}
\end{figure}
\begin{itemize}
  \item We need a stable environment to run containers
  \item We need to support automation
  \item Managing hosts should involve minimal efforts
\end{itemize}
\end{frame}

\begin{frame}[fragile]
\frametitle{Project Atomic is here!}
\begin{figure}[htp]
\centering
\includegraphics[scale=0.25]{atomic_logo.png}
\label{}
\end{figure}
\begin{verbatim}
One Project to rule them all, 
One Project to find them,
One Project to bring them all ,
and in the "containers" bind them
\end{verbatim}
\end{frame}

\begin{frame}{Introducing Atomic Host!}
\begin{figure}[htp]
\centering
\includegraphics[scale=0.45]{atom.jpg}
\label{}
\end{figure}
\begin{itemize}
  \item Minimal operating system
  \item Benefits of our favorite Enterprise Linux
  \item Robust atomic upgrades and systemd 
  \item Ready to take on cloud, virtualized or bare metal
\end{itemize}
\end{frame}

\begin{frame}{.. including rpm-ostree ..}
\begin{figure}[htp]
\centering
\includegraphics[scale=0.70]{Rpm_logo.png}
\label{}
\end{figure}
\begin{itemize}
  \item Bootable, immutable, versioned filesystem trees
  \item Composed from standard rpms
  \item Atomic upgrade and rollbacks
  \item Only /etc and /var are writable
\end{itemize}
\end{frame}

\begin{frame}{.. and Systemd ..}
\begin{figure}[htp]
\centering
\includegraphics[scale=0.45]{systemd.png}
\label{}
\end{figure}
\begin{itemize}
  \item System and service manager for Linux 
  \item Replacing the init in Centos 7
  \item Highly modular and much more powerful than sysV
  \item Check out http://0pointer.de/blog/projects/why.html
\end{itemize}
\end{frame}

\begin{frame}{.. also Introducing Cockpit..}
\begin{figure}[htp]
\centering
\includegraphics[scale=0.28]{cockpit.png}
\label{}
\end{figure}
\end{frame}

\begin{frame}{.. and lastly Kubernetes ..}
\begin{figure}[htp]
\centering
\includegraphics[scale=0.2]{kubernetes.png}
\label{}
\end{figure}
\begin{itemize}
  \item Master-slave arch
  \item Boot new containers
  \item Scalable and fault tolerant 
  \item Lots of examples and setup instructions at https://github.com/GoogleCloudPlatform/kubernetes
\end{itemize}
\end{frame}

\begin{frame}{Starting Atomic Host}
\begin{figure}[htp]
\centering
\includegraphics[scale=0.30]{boot.jpg}
\label{}
\end{figure}
\begin{itemize}
  \item Atomic host needs cloud-init data
  \item Info about the host, i.e. meta-data
  \item Info about the user, i.e. user-data
\end{itemize}
\end{frame}

\begin{frame}[fragile]
\frametitle{cloud-init data}
\begin{verbatim}
$ cat meta-data
instance-id: iid-local01;
local-hostname: myhost;

$ cat user-data
#cloud-config
password: mypassword
ssh_pwauth: True
chpasswd: { expire: False }

ssh_authorized_keys:
  - ssh-rsa ... foo@foo.com

$ genisoimage -output init.iso -volid cidata -joliet \
 -rock user-data meta-data
\end{verbatim}
\end{frame}

\begin{frame}{Demo!}
\begin{figure}[htp]
\centering
\includegraphics[scale=0.40]{demo.jpg}
\label{}
\end{figure}
\begin{itemize}
  \item Start a container.
  \item Verify that it works.
  \item Kill the container.
  \item OOOOO... Magic!
\end{itemize}
\end{frame}


\begin{frame}{Questions?}
Now is your chance :)
\end{frame}

\end{document}